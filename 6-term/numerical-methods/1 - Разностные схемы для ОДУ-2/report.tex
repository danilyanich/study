\documentclass[12pt]{article}
\usepackage[utf8x]{inputenc}
\usepackage[english,russian]{babel}
\usepackage{graphicx}
\usepackage{verbatim}
\usepackage{amsmath}
\usepackage[a4paper,margin=1.0in,footskip=0.25in]{geometry}

\makeatletter
\newcommand{\verbatimfont}[1]{\renewcommand{\verbatim@font}{\ttfamily#1}}
\author{Даниил Крачковский}


\begin{document}


	\begin{titlepage}
		\centering
		{\scshape\LARGE Численные методы математической физики \par}
		\vfill
		{\scshape\Large Лабораторная работа 1 \par}
		\vspace{1cm}
		{\huge\bfseries Разностные методы для ОДУ-2 \par}
		\vspace{2cm}
		{\Large Крачковский Даниил\par}
		5 группа \par
		\vspace{0.5cm}
		Преподаватель:\par
		Будник Анатолий Михайлович
		\vfill
		{\large \today}
	\end{titlepage}
	
	
\section*{Постановка задачи}
	Дана задача ОДУ-2:
	$$
		y''(x) + y'(x) - \dfrac{2}{cos^2(x)}y(x) = \dfrac{1}{cos^2(x)}, \qquad x \in [0, 0.5]
	$$
	$$
		y(0) = 0
	$$
	$$
		y'(0.5) = cos^2(0.5)
	$$
	Решить задачу, приведённую выше, построив разностные схемы первого и второго порядков аппроксимации и решив полученные схемы методом разностной прогонки.
\section*{Алгоритм}
	\subsection*{Формула средних прямоугольников}
	\[
		I = \sum_{k=0}^{n-1} \int_{x_k}^{x_{k+1}} f(x)dx 
	\]
	\[
		I \approx h  \sum_{k=0}^{n-1} f(\frac{x_k + x_{k+1}}{2})
	\]
	\[
		R \leq h^2 \frac{b-a}{24} \max_{x \in [a, b]} f^{\prime\prime} (x)
	\]
	\[
		h \geq \sqrt{\frac{24R}{(b-a) \max_{x \in [a, b]} f^{\prime\prime} (x)} }
	\]
	\subsection*{Формула Симпсона}
	\[
		I \approx \frac{h}{3} \left(f(x_0) + 4 \sum_{k=1}^{n/2} f(x_{2k -1}) + 2 \sum_{k=1}^{n/2 - 1} f(x_{2k}) + f(x_n)\right)
	\]
	\[
		R \leq h^4 \frac{b-a}{180} \max_{x \in [a, b]} f^{IV} (x)
	\]
	\[
		h \geq \sqrt[4]{\frac{180R}{(b-a) \max_{x \in [a, b]} f^{IV} (x)} }
	\]
	
\section{Результат}
Средних:
\verbatimfont{\small}
\begin{verbatim}
h: 0.00894427190999916
n: 111
I: 0.6909220448382232
\end{verbatim}
Симпсона:
\verbatimfont{\small}
\begin{verbatim}
h: 0.09306048591020996
n: 10
I: 0.47912965028132487
\end{verbatim}
\newpage
\section{Листинг кода}
\verbatimfont{\small}
\begin{verbatim}
import math


def f_x(x):
    p = 1.2
    return math.sqrt(p + x ** 2) / (1. + math.cos(p * x))


eps = 10. ** (-5)
max_f_2 = 3.
a = 0
b = 1

if __name__ == '__main__':
    h = math.sqrt(24.*eps/max_f_2)
    n = int(1. / h)
    I = h * sum(f_x(a + k*h) for k in range(0, n - 1))

    print('middle')
    print('h: {}'.format(h))
    print('n: {}'.format(n))
    print('I: {}'.format(I))
\end{verbatim}
\newpage
\verbatimfont{\small}
\begin{verbatim}
import math


def f_x(x):
    p = 1.2
    return math.sqrt(p + x ** 2) / (1. + math.cos(p * x))


eps = 10. ** (-5)
max_f_4 = 24.
a = 0
b = 1

if __name__ == '__main__':
    h = (180. * eps / max_f_4) ** (1./4.)
    n = int(1. / h)

    sum_1 = sum(f_x(a + (2*k - 1)*h) for k in range(1, int(n / 2)))
    sum_2 = sum(f_x(a + (2*k)*h) for k in range(1, int(n / 2) - 1))

    I = h / 3. * (f_x(a) + f_x(b) + 4 * sum_1 + 2 * sum_2)

    print('simpson')
    print('h: {}'.format(h))
    print('n: {}'.format(n))
    print('I: {}'.format(I))
\end{verbatim}
\end{document}