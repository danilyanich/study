\documentclass[12pt]{article}
\usepackage[utf8x]{inputenc}
\usepackage[english,russian]{babel}
\usepackage{graphicx}
\usepackage{verbatim}
\usepackage{amsmath}
\usepackage[a4paper,margin=1.0in,footskip=0.25in]{geometry}

\makeatletter
\newcommand{\verbatimfont}[1]{\renewcommand{\verbatim@font}{\ttfamily#1}}
\author{Даниил Крачковский}


\begin{document}


	\begin{titlepage}
		\centering
		{\scshape\LARGE Численные методы математической физики \par}
		\vfill
		{\scshape\Large Лабораторная работа 2 \par}
		\vspace{1cm}
		{\huge\bfseries Методы решения граничной задачи для ОДУ-2 \par}
		\vspace{2cm}
		{\Large Крачковский Даниил\par}
		5 группа \par
		\vspace{0.5cm}
		Преподаватель:\par
		Будник Анатолий Михайлович
		\vfill
		{\large \today}
	\end{titlepage}
	
	
\section*{Постановка задачи}
	Дана задача ОДУ-2:
	$$
		((2-x)u'(x))' - x u(x) = sin(x) - 2cos(x), \qquad x \in [0, 1]
	$$
	$$
		2u'(0) = u(0) - 1
	$$
	$$
		-u'(1) = tg(1) u(1)
	$$
	Решить данную граничную задачу ОДУ-2 методом баланса и методом Ритца.
\section*{Алгоритм}
	\subsection*{Метод баланса}
		Запишем схему, плученную данным методом, аппроксимирующую нашу задачу:
		$$
			\left( \dfrac{a_1}{h} - 1 + \dfrac{h}{2}d_0 \right) y_0 +
			\left( \dfrac{a_1}{h} \right) y_1 =
			- \left( 1 + \dfrac{h}{2}\varphi_0 \right)
		$$
		$$
			\left( \dfrac{a_i}{h^2} \right) y_{i-1} -
			\left( \dfrac{a_{i+1} + a_i}{h^2}  + d_i\right) y_i +
			\left( \dfrac{a_{i+1}}{h^2} \right) y_{i+1} =
			-\varphi_i
			\qquad 
			i = \overline{1, N-1}
		$$
		$$
			\left( \dfrac{a_N}{h} \right) y_{N-1} +
			\left( -\dfrac{a_N}{h} + tg(1) + \dfrac{h}{2} d_N \right) y_N =
			\dfrac{h}{2}\varphi_N
		$$
		где $d_i$, $\varphi_i$, $a_i$ вычичляются по следующим формулам:
		$$
			d_0 = \dfrac{2}{h} \int_0^{h/2} x dx
			\qquad
			\varphi_0 = \dfrac{2}{h} \int_0^{h/2} (2cos(x) - sin(x)) dx
		$$
		$$
			d_i = \dfrac{1}{h} \int_{x_i -h/2}^{x_i + h/2} x dx
			\qquad
			\varphi_i = \dfrac{1}{h} \int_{x_i -h/2}^{x_i + h/2} (2cos(x) - sin(x)) dx
			\qquad
			i = \overline{1, N- 1}
		$$
		$$ 
			a_i = \left( \dfrac{1}{h} \int_{x_{i-1}}^{x_i} \dfrac{dx}{2-x} \right)^{-1}
			\qquad
			i = \overline{1, N}
		$$
		$$
			d_N = \dfrac{2}{h} \int_{1-h/2}^1 x dx
			\qquad
			\varphi_N = \dfrac{2}{h} \int_{1-h/2}^1 (2cos(x) - sin(x)) dx
		$$
		Решим образовавшуюся систему методом прогонки.
	\subsection*{Метод Ритца}
		Запишем схему, плученную данным методом, аппроксимирующую нашу задачу:
		$$
			\left( \dfrac{a_i}{h^2} \right) y_{i-1} - 
			\left( \dfrac{a_{i} + a_{i+1}}{h^2} + d_i \right) y_i +
			\left( \frac{ a_{i+1}}{h^2} \right) y_{i+1} =
			-\varphi_i 
			\qquad
			i = \overline{0, N}
		$$
		где $d_i$, $\varphi_i$, $a_i$ вычичляются по следующим формулам:
		$$
			a_{i} =
			\dfrac{1}{h}
			\left(
				\int_{x_{i-1}}^{x_i} k(x)dx -
				\int_{x_{i-1}}^{x_i} q(x)(x_i-x)(x-x_{i-1})dx 
			\right) 
			\qquad 
			i = \overline{1,N}
		$$
		$$
			d_i=
			\frac{1}{h^2}
			\left(
				\int_{x_{i-1}}^{x_i} q(x)(x-x_{i-1})dx -
				\int_{x_{i}}^{x_{i+1}} q(x)(x_{i+1} - x)dx 
			\right)
			\qquad
			i = \overline{1,N-1}
		$$
		$$
			d_0=
			\frac{2}{h^2}
			\int_{0}^{h}q(x)(h-x)dx
			\qquad				
			d_N=
			\frac{2}{h^2}
			\int_{1-h}^{1}q(x)(x-1+h)dx
		$$
		$$
			\varphi_i=
			\frac{1}{h^2}
			\left(
				\int_{x_{i-1}}^{x_i}f(x)(x - x_{i-1})dx -
				\int_{x_{i}}^{x_{i+1}}f(x)(x_{i+1} - x)dx 
			\right)
			\qquad
			 i= \overline{1,N-1}
		$$
		$$
			\varphi_0=
			\frac{2}{h^2}
			\int_{0}^{h}f(x)(h - x)dx
			\qquad			
			\varphi_N=
			\frac{2}{h^2}
			\int_{1-h}^{1}
			f(x)(x - 1 + h)dx
		$$
		Решим образовавшуюся систему методом прогонки.
	
\newpage
\section{Листинг кода}
\verbatimfont{\small}
\begin{verbatim}import numpy as np
import scipy as sp
import scipy.integrate
import math as ma
import matplotlib.pyplot as plt

q_x = lambda x: x
k_x = lambda x: 2 - x
_1_k_x = lambda x: 1 / k_x(x)
f_x = lambda x: 2*ma.cos(x) - ma.sin(x)

_a, _b = 0, 1

def integrate(func, _from, to):
  return sp.integrate.quad(func, _from, to)[0]


def balance(n):
  h = (_b - _a) / n
  x = [_a + (i * h) for i in range(0, n + 1)]

  d = np.zeros(n + 1)
  d[0] = 2/h * integrate(q_x, 0, h/2)
  d[n] = 2/h * integrate(q_x, 1 - h/2, 1)

  for i in range(1, n):
    d[i] = 1/h * integrate(q_x, x[i] - h/2, x[i] + h/2)


  a = np.zeros(n + 1)

  for i in range(1, n + 1):
    a[i] = 1 / (1/h * integrate(_1_k_x, x[i - 1], x[i]))


  phi = np.zeros(n + 1)
  phi[0] = 2/h * integrate(f_x, 0, h/2)
  phi[n] = 2/h * integrate(f_x, 1 - h/2, 1)

  for i in range(1, n):
    phi[i] = 1/h * integrate(f_x, x[i] - h/2, x[i] + h/2)


  A = np.identity(n + 1)
  A[0, 0] = a[1]/h - 1 + h/2 * d[0]
  A[0, 1] = a[1]/h

  A[n, n - 1] = a[n]/h
  A[n, n] = -a[n]/h + ma.tan(1) + h/2*d[n]

  for i in range(1, n):
    A[i, i - 1] = a[i] / h**2
    A[i, i] = -((a[i + 1] + a[i]) / h**2 + d[i])
    A[i, i + 1] = a[i + 1] / h**2


  B = np.zeros(n + 1)
  B[0] = -1 - h/2 * phi[0]
  B[n] = h/2 * phi[n]

  for i in range(1, n):
    B[i] = - phi[i]

  return np.linalg.solve(A, B), x


def ritzh(n):
  h = (_b - _a) / n
  x = [_a + (i * h) for i in range(0, n + 1)]

  A = np.identity(n + 1)

  A[0, 0] = 1/h**2 * (integrate(k_x, 0, h) + integrate(lambda _x: q_x(_x) * (_x - h)**2, 0, h)) +  1
  A[n, n] = 1/h**2 * (integrate(k_x, 1 - h, 1) + integrate(lambda _x: q_x(_x) * (_x - 1 + h)**2, 1 - h, 1)) + ma.tan(1)

  for i in range(1, n):
    res = 1/h**2 * (integrate(lambda _x: q_x(_x) * (x[i + 1] - _x) * (_x - x[i]), x[i], x[i + 1]) - integrate(k_x, x[i], x[i + 1]))
    A[i, i + 1] = res
    A[i + 1, i] = res

  for i in range(1, n):
    A[i, i] = 1/h**2 * (integrate(k_x, x[i - 1], x[i + 1]) + integrate(lambda _x: q_x(_x)*(_x - x[i - 1])**2, x[i - 1], x[i]) + integrate(lambda _x: q_x(_x)*(x[i + 1] - _x)**2, x[i], x[i + 1]))


  B = np.zeros(n + 1)

  B[0] = 1/h * integrate(lambda _x: f_x(_x) * (_x - h), 0, h) + 1
  B[n] = 1/h * integrate(lambda _x: f_x(_x) * (_x - 1 + h), 1 - h, 1)

  for i in range(1, n):
    B[i] = 1/h * (integrate(lambda _x: f_x(_x) * (_x - x[i - 1]), x[i - 1], x[i]) + integrate(lambda _x: f_x(_x) * (x[i+1] - _x), x[i], x[i + 1]))

  for r in A:
    for e in r:
      print('{:15.5}'.format(e), end='')
    print()

  print(B)

  return np.linalg.solve(A, B), x


if __name__ == "__main__":

  for n in range(1, 2):
    solutionBalance, x = balance(10 * n)
    plt.plot(x, solutionBalance)

  for n in range(1, 10):
    solutionRitzh, x = ritzh(10 * n)
    plt.plot(x, solutionRitzh)

  plt.show()
\end{verbatim}
\end{document}