\documentclass[12pt]{article}
\usepackage[utf8x]{inputenc}
\usepackage[english,russian]{babel}
\usepackage{graphicx}
\usepackage{verbatim}
\usepackage{amsmath}
\usepackage[a4paper,margin=1.0in,footskip=0.25in]{geometry}

\makeatletter
\newcommand{\verbatimfont}[1]{\renewcommand{\verbatim@font}{\ttfamily#1}}
\author{Даниил Крачковский}


\begin{document}


	\begin{titlepage}
		\centering
		{\scshape\LARGE Численные методы математической физики \par}
		\vfill
		{\scshape\Large Лабораторная работа 2 \par}
		\vspace{1cm}
		{\huge\bfseries Методы решения граничной задачи для ОДУ-2 \par}
		\vspace{2cm}
		{\Large Крачковский Даниил\par}
		5 группа \par
		\vspace{0.5cm}
		Преподаватель:\par
		Будник Анатолий Михайлович
		\vfill
		{\large \today}
	\end{titlepage}
	
	
\section*{Постановка задачи}
	Дана задача ОДУ-2:
	$$
		((2-x)u'(x))' - x u(x) = sin(x) - 2cos(x), \qquad x \in [0, 1]
	$$
	$$
		2u'(0) = u(0) - 1
	$$
	$$
		-u'(1) = tg(1) u(1)
	$$
	Решить данную граничную задачу ОДУ-2 методом баланса и методом Ритца.
\section*{Алгоритм}
	\subsection*{Метод баланса}
		Запишем схему, плученную данным методом, аппроксимирующую нашу задачу:
		$$
			\left( \dfrac{a_1}{h} - 1 + \dfrac{h}{2}d_0 \right) y_0 +
			\left( \dfrac{a_1}{h} \right) y_1 =
			- \left( 1 + \dfrac{h}{2}\varphi_0 \right)
		$$
		$$
			\left( \dfrac{a_i}{h^2} \right) y_{i-1} -
			\left( \dfrac{a_{i+1} + a_i}{h^2}  + d_i\right) y_i +
			\left( \dfrac{a_{i+1}}{h^2} \right) y_{i+1} =
			-\varphi_i
			\qquad 
			i = \overline{1, N-1}
		$$
		$$
			\left( \dfrac{a_N}{h} \right) y_{N-1} +
			\left( -\dfrac{a_N}{h} + tg(1) + \dfrac{h}{2} d_N \right) y_N =
			\dfrac{h}{2}\varphi_N
		$$
		где $d_i$, $\varphi_i$, $a_i$ вычичляются по следующим формулам:
		$$
			d_0 = \dfrac{2}{h} \int_0^{h/2} x dx
			\qquad
			\varphi_0 = \dfrac{2}{h} \int_0^{h/2} (2cos(x) - sin(x)) dx
		$$
		$$
			d_i = \dfrac{1}{h} \int_{x_i -h/2}^{x_i + h/2} x dx
			\qquad
			\varphi_i = \dfrac{1}{h} \int_{x_i -h/2}^{x_i + h/2} (2cos(x) - sin(x)) dx
			\qquad
			i = \overline{1, N- 1}
		$$
		$$ 
			a_i = \left( \dfrac{1}{h} \int_{x_{i-1}}^{x_i} \dfrac{dx}{2-x} \right)^{-1}
			\qquad
			i = \overline{1, N}
		$$
		$$
			d_N = \dfrac{2}{h} \int_{1-h/2}^1 x dx
			\qquad
			\varphi_N = \dfrac{2}{h} \int_{1-h/2}^1 (2cos(x) - sin(x)) dx
		$$
		Решим образовавшуюся систему методом прогонки.
	\subsection*{Метод Ритца}
		Запишем схему, плученную данным методом, аппроксимирующую нашу задачу:
		$$
			\left( \dfrac{a_i}{h^2} \right) y_{i-1} - 
			\left( \dfrac{a_{i} + a_{i+1}}{h^2} + d_i \right) y_i +
			\left( \frac{ a_{i+1}}{h^2} \right) y_{i+1} =
			-\varphi_i 
			\qquad
			i = \overline{0, N}
		$$
		где $d_i$, $\varphi_i$, $a_i$ вычичляются по следующим формулам:
		$$
			a_{i} =
			\dfrac{1}{h}
			\left(
				\int_{x_{i-1}}^{x_i} k(x)dx -
				\int_{x_{i-1}}^{x_i} q(x)(x_i-x)(x-x_{i-1})dx 
			\right) 
			\qquad 
			i = \overline{1,N}
		$$
		$$
			d_i=
			\frac{1}{h^2}
			\left(
				\int_{x_{i-1}}^{x_i} q(x)(x-x_{i-1})dx -
				\int_{x_{i}}^{x_{i+1}} q(x)(x_{i+1} - x)dx 
			\right)
			\qquad
			i = \overline{1,N-1}
		$$
		$$
			d_0=
			\frac{2}{h^2}
			\int_{0}^{h}q(x)(h-x)dx
			\qquad				
			d_N=
			\frac{2}{h^2}
			\int_{1-h}^{1}q(x)(x-1+h)dx
		$$
		$$
			\varphi_i=
			\frac{1}{h^2}
			\left(
				\int_{x_{i-1}}^{x_i}f(x)(x - x_{i-1})dx -
				\int_{x_{i}}^{x_{i+1}}f(x)(x_{i+1} - x)dx 
			\right)
			\qquad
			 i= \overline{1,N-1}
		$$
		$$
			\varphi_0=
			\frac{2}{h^2}
			\int_{0}^{h}f(x)(h - x)dx
			\qquad			
			\varphi_N=
			\frac{2}{h^2}
			\int_{1-h}^{1}
			f(x)(x - 1 + h)dx
		$$
		Решим образовавшуюся систему методом прогонки.
	
\section*{Результат}
Средних:
\verbatimfont{\small}
\begin{verbatim}
h: 0.00894427190999916
n: 111
I: 0.6909220448382232
\end{verbatim}
Симпсона:
\verbatimfont{\small}
\begin{verbatim}
h: 0.09306048591020996
n: 10
I: 0.47912965028132487
\end{verbatim}
\newpage
\section{Листинг кода}
\verbatimfont{\small}
\begin{verbatim}
import math


def f_x(x):
    p = 1.2
    return math.sqrt(p + x ** 2) / (1. + math.cos(p * x))


eps = 10. ** (-5)
max_f_2 = 3.
a = 0
b = 1

if __name__ == '__main__':
    h = math.sqrt(24.*eps/max_f_2)
    n = int(1. / h)
    I = h * sum(f_x(a + k*h) for k in range(0, n - 1))

    print('middle')
    print('h: {}'.format(h))
    print('n: {}'.format(n))
    print('I: {}'.format(I))
\end{verbatim}
\newpage
\verbatimfont{\small}
\begin{verbatim}
import math


def f_x(x):
    p = 1.2
    return math.sqrt(p + x ** 2) / (1. + math.cos(p * x))


eps = 10. ** (-5)
max_f_4 = 24.
a = 0
b = 1

if __name__ == '__main__':
    h = (180. * eps / max_f_4) ** (1./4.)
    n = int(1. / h)

    sum_1 = sum(f_x(a + (2*k - 1)*h) for k in range(1, int(n / 2)))
    sum_2 = sum(f_x(a + (2*k)*h) for k in range(1, int(n / 2) - 1))

    I = h / 3. * (f_x(a) + f_x(b) + 4 * sum_1 + 2 * sum_2)

    print('simpson')
    print('h: {}'.format(h))
    print('n: {}'.format(n))
    print('I: {}'.format(I))
\end{verbatim}
\end{document}